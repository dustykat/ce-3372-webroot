\documentclass[11pt]{article}
\usepackage{geometry}                % See geometry.pdf to learn the layout options. There are lots.
\geometry{letterpaper}                   % ... or a4paper or a5paper or ... 
%\geometry{landscape}                % Activate for for rotated page geometry
\usepackage[parfill]{parskip}    % Activate to begin paragraphs with an empty line rather than an indent
\usepackage{daves,fancyhdr,natbib,graphicx,dcolumn,amsmath,lastpage,url}
\usepackage{amsmath,amssymb,epstopdf,longtable}
\usepackage{paralist}  % need to properly formulate standard answer blocks
\usepackage[final]{pdfpages}
\DeclareGraphicsRule{.tif}{png}{.png}{`convert #1 `dirname #1`/`basename #1 .tif`.png}
\pagestyle{fancy}
\lhead{CE 3372 Water Systems Design; Exam 1}
\rhead{Name:\_\_\_\_\_\_\_\_\_\_\_\_\_\_\_\_\_\_\_\_\_\_\_\_\_\_\_\_\_\_\_\_\_\_}
\lfoot{SPRING 2017 A.}
\cfoot{}
\rfoot{Page \thepage\ of \pageref{LastPage}}
\renewcommand\headrulewidth{0pt}
\newcommand\tab[1][1cm]{\hspace*{#1}}



\begin{document}
%%%%%%%%%%%%%%%%%%%%%%%%%%%%%%%%%%%
%\begingroup
%\begin{center}
%{\textbf{{ CE 3372 Water Systems Design Exam 1 }} }
%\end{center}
%\endgroup

%%%%%%%%%%%%%%%%%%%%%%%%%%%

\begin{enumerate}
\item One acre is a square with sides of length of 208.71 feet.  How many square feet in an acre?
~\\
~\\
\item 1 inch is $~\approx 25.4$ millimeters.  What is the diameter of a six-inch pipe, in millimeters?
~\\
~\\
\item 1 meter is $~\approx 3.28$ feet.   There are 5280 feet in a mile.   How many meters in one mile?
~\\
~\\
~\\
~\\
\item One hectare is a square with sides of length 100 meters.  How many acres in one hectare?
~\\
~\\
~\\
~\\
~\\
\begin{figure}[h!] %  figure placement: here, top, bottom, or page
   \centering
   \includegraphics[height=2.5in]{greek-storm-drain.pdf} 
   \caption{Photograph of ancient storm drain}
   \label{fig:greek-storm-drain}
\end{figure}
\item Figure \ref{fig:greek-storm-drain} is a photograph of a
\begin{enumerate}[(a)]
\item Water use system
\item Water control system
\item Environmental restoration system
\item W{\"a}sserb{\"u}rger system
\end{enumerate}

\clearpage
\begin{figure}[h!] %  figure placement: here, top, bottom, or page
   \centering
   \includegraphics[height=4in]{Tank.jpg} 
   \caption{Square detention pond}
   \label{fig:Tank}
\end{figure}
\item Figure \ref{fig:Tank} is a square (plan view) detention pond.  The initial depth in the 1 acre (plan view) ($L=208.71\text{~foot}$) is $Z=10\text{~feet}$.  
What is the average outflow rate in cubic-feet-per-second, if the pond drains through the hole in the side in 24 hours?
\clearpage
\item Figure \ref{fig:SewerDrawing} is a portion of an engineering drawing of a gravity-flow wastewater conduit. 
\begin{figure}[h!] %  figure placement: here, top, bottom, or page
   \centering
   \includegraphics[height=6.2in]{SewerDrawing.jpg} 
   \caption{Engineering drawing of sanitary sewer system (display is rotated)}
   \label{fig:SewerDrawing}
\end{figure}

\begin{enumerate}[(a)]
\item What object is located at station 10+38.48?
\item What is the invert elevation of the pipe at station 13+00?
\item What is the diameter of the pipe in inches?
\item What direction is sewage intended to flow?
\end{enumerate}
\clearpage
%%%%%%%%%%%%%%%%%%%%
% Hazen-Williams Application (HIPS) %
%%%%%%%%%%%%%%%%%%%%
\item Equation \ref{eqn:hazen-williams} is the  Hazen-Williams discharge model for U.S. Customary units.
\begin{equation}
Q = 1.318~A~C_h~R^{0.63}~S^{0.54}
\label{eqn:hazen-williams}
\end{equation}

where;\\~\\
\tab $Q$ is the discharge in $ft^3/sec$;\\
\tab $A$ is the cross section area of pipe in $ft^2$ ($A = \frac{\pi D^2}{4}$; $D$ is the pipe diameter.);\\
\tab $C_h$ is the Hazen-Williams friction coefficient (depends on pipe material);\\
\tab $R $ is the hydraulic radius in $ft$;  ($R=\frac{A}{P_W}$)\\
\tab $P_W$ is the wetted perimeter; \\
\tab $S$ is the slope of the energy grade line ($\frac{h_f}{L}$); and \\
\tab $L$ is the length of pipe.

For a 12,000 foot long, 6-foot diameter steel pipe, determine:
\begin{enumerate}[(a)]
\item The wetted perimeter, $P_W$, in feet. \\~\\~\\
\item The cross sectional area of flow, $A$, in square feet. \\~\\~\\
\item The hydraulic radius, $R$, in feet. \\~\\~\\~\\
\item The head loss in the pipe, if $C_h~=150$, and the discharge is $Q=295\text{~cubic feet per second}$

\end{enumerate}


\clearpage
%%%%%%%%%%%%%%%%%%%
%   Jain Application (HIPS, MSMD)  %
%%%%%%%%%%%%%%%%%%%
\item 
Equation \ref{eqn:flow-jain} is an explicit formula (based on the Darcy-Weisbach head loss model and the Colebrook-White frictional loss equation)for estimating discharge from head loss and material properties. %\citep{jain1976}.
\begin{equation}
Q=-2.22D^{5/2} \times \sqrt{gh_f/L}\times[log_{10} (\frac{k_s}{3.7D} + \frac{1.78\nu}{D^{3/2}\sqrt{gh_f/L}} )]
\label{eqn:flow-jain}
\end{equation}

where;\\~\\
\tab $Q$ is the discharge in $L^3/T$;\\
\tab $D$ is the pipe diameter; \\
\tab $h_f$ is the head loss in the pipe; \\
\tab $g$ is the gravitational acceleration constant; \\
\tab $L$ is the length of pipe; \\
\tab $k_s$ is the pipe roughness height; \\
\tab $\nu$ is the kinematic viscosity of liquid in the pipe; \\ 

Estimate the discharge in cubic-feet-per-second for the 3.2 mile long, 24-inch diameter, ductile iron pipe connecting points A and B as depicted Figure \ref{fig:PipePressureProblem}.
The sand roughness height of ductile iron is $8.5\times10^{-4}~ft$.
The 50$^o$F water has kinematic viscosity of $1.45\times10^{-5}~ft^2/s$.
Point A is 28 feet higher in elevation than point B.  The pressure at point B is 21 pounds per square-inch (psi) greater than the pressure at point A.
(Show your work on the next page)

\begin{figure}[h!] %  figure placement: here, top, bottom, or page
   \centering
   \includegraphics[height=3in]{PipePressureProblem.jpg} 
   \caption{Pipeline Schematic}
   \label{fig:PipePressureProblem}
\end{figure}
\clearpage
Problem 9 (Continued)\\
~\newline


\clearpage
%%%%%%%%%%%%%%%%%%
% Tehachapi Pipelines YOYO %%%
%%%%%%%%%%%%%%%%%%
\item Figure \ref{fig:parallelpipes} is an aerial image of a parallel pipeline system in California.   

\begin{figure}[htbp] %  figure placement: here, top, bottom, or page
   \centering
   \includegraphics[height=5in]{parallelpipes.jpg} 
   \caption{Parallel Pipeline System}
   \label{fig:parallelpipes}
\end{figure}

The left pipeline is a 96-inch diameter steel pipe, whereas the right pipeline is a 108-inch diameter steel pipe.  
Water at 50$^o$F has kinematic viscosity of $1.45\times10^{-5}~ft^2/s$.   
The sand roughness of ductile iron is $1.64\times10^{-4}~ft$.   
If the head difference for the one-mile long pipelines between the thrust blocks is 100 feet, determine the discharge in each pipe in cubic-feet-per-second.
(Show your work on the next page)
\clearpage
Problem 10 (continued)
\clearpage

%%%%%%%%%%%%%%%%%%%%%%%%%%%%%
%%%%%%%%%%%%%%%%%%%%%%%%%%%%%%%%%%%%%%%%%%%%%%%%%%
Figure \ref{fig:wheatstone_bridge} is a pipe network with the following properties:
\begin{figure}[h!] %  figure placement: here, top, bottom, or page
\centering
   \includegraphics[width=5.1in]{wheatstone_bridge.jpg}
   \caption{Pipe network}
   \label{fig:wheatstone_bridge} 
\end{figure}

\begin{table}[h!]
%\footnotesize
   \centering
   \caption{Network properties for Figure \ref{fig:wheatstone_bridge}}
\begin{tabular}{lll}
\hline
Node&Demand (cfs)&  Elevation(ft) \\
\hline
A & -0.60 & 0.00 \\
B &  0.15   & 0.00 \\
C & 0.15 & 0.00 \\
D & 0.30 & 0.00 \\
\hline
Pipe& Length (ft)&  Diameter (ft)\\
\hline
1 & 1000 & 3/12 \\
2 & 1000 & 3/12\\
3 & 1000 & 3/12\\
4 & 1000 & 3/12\\
5 & 1400 & 3/12\\
\end{tabular}
\label{tab:wheatstone1}
\normalsize
\end{table}
\clearpage

%\item Referring to Figure \ref{fig:wheatstone_bridge}, the discharge in pipe 5 is closest to
%%standard answer set
%\begin{enumerate}[(A)]
%\item 0.00 cfs, from Node B to Node C  %Correct
%\item 0.15 cfs, from Node C to Node B
%\item 0.66 cfs, from Node C to Node B
%\item 0.66 cfs, from Node B to Node C
%\end{enumerate}


%%%%%%%%%%%%%%%%%%%%%%%%%%%%

\item Referring to Figure \ref{fig:wheatstone_bridge}, and Table \ref{tab:wheatstone1} the flow distribution is:
%standard answer set
\begin{enumerate}[(A)]
\item $[Q_1,~Q_2,~Q_3,~Q_4,~Q_5~]~=$[0.30,0.15,0.15,0.30,0.00] CFS
\item $[Q_1,~Q_2,~Q_3,~Q_4,~Q_5~]~=$[0.30,0.15,0.15,0.30,0.30] CFS
\item $[Q_1,~Q_2,~Q_3,~Q_4,~Q_5~]~=$[0.30,0.15,0.15,0.00,0.50] CFS
\item $[Q_1,~Q_2,~Q_3,~Q_4,~Q_5~]~=$[0.30,-0.15,-0.15,0.30,-0.60] CFS
\end{enumerate}
\item Referring to Figure \ref{fig:wheatstone_bridge}, assuming the average friction factor is $0.018$, the head loss, in feet, from Node A to Node C is closest to
%standard answer set
\begin{enumerate}[(A)]
\item 12 feet 
\item 25 feet
\item 50 feet
\item 75 feet
\end{enumerate}
\item Referring to Figure \ref{fig:wheatstone_bridge}, if the demands at all nodes are those in Table \ref{tab:wheatstone1}, and pipe 2 is decreased to a diameter of 2/12, the discharge in pipe 5 is closest to
%standard answer set
\begin{enumerate}[(A)]
\item 0.00 cfs, from Node C to Node B  %Correct
\item 0.15 cfs, from Node B to Node C
\item 0.30 cfs, from Node C to Node B
\item 0.60 cfs, from Node B to Node C
\end{enumerate}
\clearpage
%%%%%%%%%%%%%%%%%%%%%%%%%%%%
\item Water is to be pumped at a rate of 70 liters per second in a 1-kilometer meter long, 200 millimeter diameter pipeline between two reservoirs with an elevation difference of 20 meters.  The roughness height of the steel pipe is 0.045 millimeters.   

The Reynolds number for water is computed from
\begin{equation}
{Re}=\frac{VD}{\nu}
\end{equation}
The kinematic viscosity of water in the system is 
\begin{equation}
{\nu}=1\times~10^{-6} \text{m}^2/\text{s}
\end{equation}
The Jain equation (Jain, 1976) that directly computes friction factor from Reynolds number, diameter, and roughness is 
\begin{equation}
f = \frac{0.25}{[log(\frac{k_s}{3.7D} + \frac{5.74}{Re^{0.9}})]^2}
\end{equation}
The Darcy-Weisbach head loss equation (for pipe losses) is
\begin{equation}
h_{loss}=f\frac{L}{D}\frac{V^2}{2g}
\end{equation}

Using the description and these equations 
\begin{enumerate}[a)]
\item Sketch the system -- show the two reservoirs and the pump on the sketch.\\
~\\
~\\
~\\
~\\
~\\
~\\
~\\
~\\
~\\
\item Convert the pipe diameter into meters (you will need this value below).\\
~\\
~\\
~\\

\item Convert the flow rate into cubic-meters-per-second (you will need this value below).\\
~\\
~\\
~\\

\item Determine the pipeline velocity (in meters per second).\\
~\\
~\\
~\\
\item Compute the Reynolds number for the system.\\
~\\
~\\
~\\
\item Compute the friction factor from the Jain equation.\\
~\\
~\\
~\\

\item What is the pipe head loss in the system at 70 liters per second?\\
~\\
~\\
~\\
\item What is the static lift (in meters of head)?
~\\
~\\
\item What is the sum of the frictional loss and the static lift (in meters of head)?
~\\
~\\
~\\
~\\
Select a pump from one of the four on Figure \ref{fig:pumps} below.  Indicate the operating point on the pump you select.
~\\
\item Which pump and operating impeller speed did you choose?
~\\
~\\
\item Estimate the required NPSH for the pump you choose.
~\\
~\\
\end{enumerate}

\begin{figure}[h]
\centering
\includegraphics[width=0.5\textwidth]{pump1.pdf}%
\includegraphics[width=0.5\textwidth]{pump2.pdf}\\%[0.5cm]%can add fill space here if you want
\includegraphics[width=0.5\textwidth]{pump3.pdf}%
\includegraphics[width=0.5\textwidth]{pump4.pdf}%
\caption{Pump curves for four different pumps.  Each pump has five different impeller speeds for 20 possible different combinations.}
\label{fig:pumps}
\end{figure}

\clearpage

\item Critique the memorandum in Figure \ref{fig:memorandum}.  Mark directly on the figure -- you are to identify the grammar, spelling, punctuation and format errors and state the required corrections.

\begin{figure}[h!] %  figure placement: here, top, bottom, or page
\centering
   \includegraphics[height=7.3in]{Exam-1-Memo.jpg}
   \caption{Pipe network}
   \label{fig:memorandum} 
\end{figure}

\end{enumerate}

\end{document}  

%%%%%%%%%%%%%%%%%%%%%%%%%%%%%%%%%%%%%%%%%%%%%%%%%
%%%%%%%PROBLEM 1 %%%%%%%%%%%%%%%%%%%%%%%%%%%%%%%%%%%
%%%%%%%%%%%%%%%%%%%%%%%%%%%%%%%%%%%%%%%%%%%%%%%%%

\item A circular, 60-inch diameter, reinforced concrete sewer pipe ($n = 0.013$ )carries 50 MGD of wastewater to a lift station wet well.   Average slope along the flow path is 1.0\%.
%about 50 MGD%
\begin{enumerate} 
\item	Sketch the cross section, indicate the pipe diameter.
~\\
~\\
~\\
~\\
~\\
~\\
~\\
~\\
~\\
~\\
~\\
~\\
~\\
~\\
\item For the conditions in the problem statement, what is the flow rate in cubic feet per second?
~\\
~\\
\item What is the diameter of the pipe, in feet?
~\\
~\\


\item Use Manning's equation ($ Q = \frac{1.49}{n} A R^{(2/3)} S^{(1/2)} $) and determine the \textbf{pipe-full} discharge in cubic feet per second?
~\\
~\\
~\\
~\\
~\\
~\\
\clearpage
\item What is the pipe-full discharge ($Q_{f}$) in million gallons per day (MGD)?
~\\
~\\
~\\
\item Compute the ratio of actual flow ($\frac{Q}{Q_{f}}$) to full pipe flow.
~\\
~\\
~\\
\item What is the ratio of depth of actual flow to full flow ($\frac{d}{D}$) using the hydraulic element chart in Figure \ref{fig:hydraulic-elements}?  Use the highlighted curve.
~\\
~\\
~\\
\begin{figure}[ht!] %  figure placement: here, top, bottom, or page
\centering
   \includegraphics[width=4.5in]{hydraulic-elements.pdf}
   \caption{Hydraulic Elements Chart}
   \label{fig:hydraulic-elements} 
\end{figure}
\clearpage
\item What is the depth of actual flow in feet?
~\\
~\\
~\\
\item What is the depth of actual flow in inches?
~\\
~\\
~\\
\item Modify your sketch to include the water surface position and the approximate flow depth.
~\\
\item Is this portion of sewer close to surcharging?
~\\
\end{enumerate}

\clearpage
%%%%%%%%%%%%%%%%%%%%%%%%%%%%%%%%%%%%%%%%%%%%%%%%%
%%%%%%%%%%%PROBLEM 2 %%%%%%%%%%%%%%%%%%%%%%%%%%%%%%%
%%%%%%%%%%%%%%%%%%%%%%%%%%%%%%%%%%%%%%%%%%%%%%%%%
\item The questions on the following page refer to Figure \ref{fig:sewer} . 

\begin{figure}[ht!] %  figure placement: here, top, bottom, or page
\centering
   \includegraphics[width=5.5in]{sewer.pdf}
   \caption{Plan and profile of a storm sewer system}
   \label{fig:sewer} 
\end{figure}
\begin{enumerate}
\clearpage

\item Is the design flow in the drawing from left-to-right, or right-to-left?
~\\
~\\
\item The drawing indicates a drop at a junction box.  What is the bottom elevation of the junction box indicated on the drawing?
~\\
~\\
\item What is the diameter of the conduits indicated on the drawing?
~\\
~\\
\item What is the slope (in percent) of the storm sewer conduits?
~\\
~\\
\item Relative to the drop, what is the flow-line (invert) elevation of the left-most sewer pipe?
~\\
~\\
\item Relative to the drop, what is the flow-line (invert) elevation of the right-most sewer pipe?
~\\
~\\
\item Relative to the drop, what is the soffit (crown) elevation of the left-most sewer pipe?
~\\
~\\
\item Relative to the drop, what is the soffit (crown) elevation of the right-most sewer pipe?
~\\
~\\
\end{enumerate}
\clearpage
\item Figure \ref{fig:diameter-increase} is a plan-and profile of a portion of a storm sewer system.  Answer the questions on the following page.
\begin{figure}[ht!] %  figure placement: here, top, bottom, or page
\centering
   \includegraphics[width=5in]{diameter-increase.pdf}
   \caption{Plan and profile of a storm sewer system}
   \label{fig:diameter-increase} 
\end{figure}
\clearpage
\begin{enumerate}
\item What is the diameter of the conduit on the right-side of the drawing?
~\\
\item What is the slope of the conduit (in percent) on the right-side of the drawing?
~\\
\item What is the diameter of the conduit on the left-side of the drawing?
~\\
\item What is the slope of the conduit (in percent) on the right-side of the drawing?
~\\
\item What is the soffit (crown) elevation of the right-most sewer pipe?
~\\
\item What is the soffit (crown) elevation of the left-most sewer pipe?
~\\
\item What is the flow-line (invert) elevation of the right-most sewer pipe?
~\\
\item What is the flow-line (invert) elevation of the left-most sewer pipe?
~\\
\item Explain, using sketches as necessary, why sewer soffit elevations should match (or drop in the flow direction) at a junction when the pipe diameter increases moving downstream. 
\end{enumerate}

%%%%%%%%%%%%%%%%%%
% EPA-NET MSMD %%%%%%%%
%%%%%%%%%%%%%%%%%%
\item An EPA-NET simulation produced the  "full report" listed below.   
\begin{verbatim}
  **********************************************************************
  Link - Node Table:
  ----------------------------------------------------------------------
  Link           Start          End                Length  Diameter
  ID             Node           Node                   ft        in
  ----------------------------------------------------------------------
  1              2              3                    1000        12
  2              3              4                    1000        12
  3              2              5                    1000        12
  4              3              6                    1000        12
  5              5              6                    1000        12
  6              6              4                    1000        12
  7              1              2                    1000        12
  Node Results:
  ----------------------------------------------------------------------
  Node                Demand      Head  Pressure   Quality
  ID                     CFS        ft       psi          
  ----------------------------------------------------------------------
  2                     0.00     76.94     33.34      0.00
  3                     0.00     75.55     32.74      0.00
  4                     3.00     74.51     32.28      0.00
  5                     0.00     76.23     33.03      0.00
  6                     0.00     75.51     32.72      0.00
  1                    -3.00    100.00      0.00      0.00 Reservoir
  Link Results:
  ----------------------------------------------------------------------
  Link                  Flow  VelocityUnit Headloss    Status
  ID                     CFS       fps    ft/Kft
  ----------------------------------------------------------------------
  1                     1.76      2.25      1.39      Open
  2                     1.51      1.93      1.04      Open
  3                     1.24      1.57      0.71      Open
  4                     0.25      0.32      0.04      Open
  5                     1.24      1.57      0.71      Open
  6                     1.49      1.89      1.01      Open
  7                     3.00      3.82     23.06      Open
\end{verbatim}
\clearpage
Using the information contained in the EPA-NET report on the previous page:
\begin{enumerate}[a)]
\item How many pipes are in the network?
~\\
\item How many junctions (nodes) are in the network?
~\\
\item How many reservoirs/tanks are in the network?
~\\
\item Sketch the network, label the pipes, junctions, and reservoirs.   Indicate any demands at nodes.   Indicate the flow rates and flow directions on your sketch.
\end{enumerate}

\item  An EPA-NET simulation model for a reservoir-pump-network was constructed and operated for four (4) different operational scenarios.   Figure \ref{fig:epa-net-map} is a depiction of the network.   The numbers next to the nodes are Node\_ID values in the reports that follow, and the numbers next to the pipes are the Link\_ID values.  The network is supplied from a reservoir through a booster pump, both are depicted on Figure \ref{fig:epa-net-map}. 

\begin{figure}[h!] %  figure placement: here, top, bottom, or page
\centering
   \includegraphics[width=3in]{epa-net-map.pdf}
   \caption{EPA-NET system topology.}
   \label{fig:epa-net-map} 
\end{figure}

Figure \ref{fig:epanet1} is the a portion of the summary report for simulation 1.   
Figure \ref{fig:epanet2} is the a portion of the summary report for simulation 2.  
Figure \ref{fig:epanet3} is the a portion of the summary report for simulation 3.  
Figure \ref{fig:epanet4} is the a portion of the summary report for simulation 4.

These four simulation represent different demand scenarios for the same system.



Interpret these reports, to answer the following questions:

\begin{enumerate}
\item Complete the table below.  $Q_{pump}$ is the discharge in liters-per-second through the pump station, $H_{Supply}$ is the head at the supply reservoir,  $H_{Node2}$ is the head at Node 2, and $\Delta H_{pump}$ is the added head supplied by the pump.
% Requires the booktabs if the memoir class is not being used
\begin{table}[htbp]
   \centering
      \caption{Pump Discharge and Supplied Head}
   \begin{tabular}{p{1in} p{1in} p{1in} p{1in} p{1in} } % Column formatting, @{} suppresses leading/trailing space
Simulation \# & $Q_{pump}$ & $H_{Supply}$ & $H_{Node2}$ & $\Delta H_{pump}$ \\
\hline
\hline
~~1 & ~ &~ & ~ & ~ \\
~ & ~ &~ & ~ & ~ \\
\hline
~~2 & ~ &~ & ~ & ~ \\
~ & ~ &~ & ~ & ~ \\
\hline
~~3 & ~ &~ & ~ & ~\\
~ & ~ &~ & ~ & ~ \\
\hline
~~4 & ~ &~ & ~ & ~ \\
~ & ~ &~ & ~ & ~ \\
\hline
   \end{tabular}
   \label{tab:pump-curve}
\end{table}

\item Complete the table below.  $Q_{pump}$ is the discharge in liters-per-second through the pump station, $\Delta H_{Node 2 -to- 5}$ is head loss in the system from Node 2 to Node 5.
\begin{table}[htbp]
   \centering
      \caption{System Discharge and Head Loss}
   \begin{tabular}{p{1in} p{1in} p{1in} p{1in} p{1in} } % Column formatting, @{} suppresses leading/trailing space
Simulation \# & $Q_{pump}$ & $H_{Node2}$ & $H_{Node5}$ & $\Delta H_{Node 2 -to- 5}$ \\
\hline
\hline
~~1 & ~ &~ & ~ & ~ \\
~ & ~ &~ & ~ & ~ \\
\hline
~~2 & ~ &~ & ~ & ~ \\
~ & ~ &~ & ~ & ~ \\
\hline
~~3 & ~ &~ & ~ & ~\\
~ & ~ &~ & ~ & ~ \\
\hline
~~4 & ~ &~ & ~ & ~ \\
~ & ~ &~ & ~ & ~ \\
\hline
   \end{tabular}
   \label{tab:system-curve}
\end{table}



\item If the pump performance curve has the mathematical structure: ~\\
$H_{pump} = H_{shutoff} - K_{pipe} \times Q^2$, estimate the values of $H_{shutoff}$  and $K_{pipe}$.
\\
\\
\\
\\
\\
\\
\\
\\
\\
\\
\\
\\
\\
\item If the system frictional loss curve has the mathematical structure:
 $H_{pipe}= K_{loss} \times Q^2$, estimate the value of $K_{loss}$

\clearpage
\item What effect would removing the pipe joining nodes 3 and 4 have on the system performance?   Explain your reasoning.
\\
\\
\\
\\
\\
\\
\\
\\
\\
\\
\\
\\
\\
\item Estimate the flow distribution and head losses the the system if the the pipe joining nodes 3 and 4 are removed, and the pipe joining node 4 and 5 is removed if the nodal demands are the same as SIMULATION  2.
\end{enumerate} 

%%% YOUR ON YOUR OWN %%%%%%%%%%%%%%%%%%%%
\begin{figure}[ht!] %  figure placement: here, top, bottom, or page
\centering

\begin{verbatim}
  Page 1                                            10/4/2010 2:27:47 PM
  **********************************************************************
  *                             E P A N E T                            *
  *                     Hydraulic and Water Quality                    *
  *                     Analysis for Pipe Networks                     *
  *                           Version 2.0                              *
  **********************************************************************
  Input File: SIMULATION #1
Link - Node Table:
  ----------------------------------------------------------------------
  Link           Start          End                Length  Diameter
  ID             Node           Node                    m        mm
  ----------------------------------------------------------------------
  1              2              3                    1000       124
  2              3              5                    1000       124
  3              2              4                    1000       124
  4              4              5                    1000       124
  5              3              4                    1400       124
  7              6              2                    #N/A      #N/A Pump
Node Results:
  ----------------------------------------------------------------------
  Node                Demand      Head  Pressure   Quality
  ID                     LPS         m         m          
  ----------------------------------------------------------------------
  2                     0.00     20.00     20.00      0.00
  3                     0.00     20.00     20.00      0.00
  4                     0.00     20.00     20.00      0.00
  5                     0.00     20.00     20.00      0.00
  6                     0.00      0.00      0.00      0.00 Reservoir
Link Results:
  ----------------------------------------------------------------------
  Link                  Flow  VelocityUnit Headloss    Status
  ID                     LPS       m/s      m/km
  ----------------------------------------------------------------------
  1                     0.00      0.00      0.00      Open
  2                     0.00      0.00      0.00      Open
  3                     0.00      0.00      0.00      Open
  4                     0.00      0.00      0.00      Open
  5                     0.00      0.00      0.00      Open
  7                     0.00      0.00    -20.00      Open Pump
  \end{verbatim}
     \caption{EPA-NET Summary Report, Simulation \#1}
   \label{fig:epanet1} 
\end{figure}


\begin{figure}[ht!] %  figure placement: here, top, bottom, or page
\centering
\begin{verbatim}
 Page 1                                            10/4/2010 2:28:15 PM
  **********************************************************************
  *                             E P A N E T                            *
  *                     Hydraulic and Water Quality                    *
  *                     Analysis for Pipe Networks                     *
  *                           Version 2.0                              *
  **********************************************************************
  Input File: SIMULATION 2
  Link - Node Table:
  ----------------------------------------------------------------------
  Link           Start          End                Length  Diameter
  ID             Node           Node                    m        mm
  ----------------------------------------------------------------------
  1              2              3                    1000       124
  2              3              5                    1000       124
  3              2              4                    1000       124
  4              4              5                    1000       124
  5              3              4                    1400       124
  7              6              2                    #N/A      #N/A Pump
  Node Results:
  ----------------------------------------------------------------------
  Node                Demand      Head  Pressure   Quality
  ID                     LPS         m         m          
  ----------------------------------------------------------------------
  2                     0.00     19.28     19.28      0.00
  3                     1.00     19.03     19.03      0.00
  4                     1.00     19.03     19.03      0.00
  5                     1.00     18.99     18.99      0.00
  6                    -3.00      0.00      0.00      0.00 Reservoir                                                              
  Link Results:
  ----------------------------------------------------------------------
  Link                  Flow  VelocityUnit Headloss    Status
  ID                     LPS       m/s      m/km
  ----------------------------------------------------------------------
  1                     1.50      0.12      0.25      Open
  2                     0.50      0.04      0.03      Open
  3                     1.50      0.12      0.25      Open
  4                     0.50      0.04      0.03      Open
  5                     0.00      0.00      0.00      Open
  7                     3.00      0.00    -19.28      Open Pump
  \end{verbatim}
     \caption{EPA-NET Summary Report, Simulation \#2}
   \label{fig:epanet2} 
\end{figure}

\begin{figure}[ht!] %  figure placement: here, top, bottom, or page
\centering

\begin{verbatim}
   Page 1                                            10/4/2010 2:29:00 PM
  **********************************************************************
  *                             E P A N E T                            *
  *                     Hydraulic and Water Quality                    *
  *                     Analysis for Pipe Networks                     *
  *                           Version 2.0                              *
  **********************************************************************
  Input File: SIMULATION 4
 Link - Node Table:
  ----------------------------------------------------------------------
  Link           Start          End                Length  Diameter
  ID             Node           Node                    m        mm
  ----------------------------------------------------------------------
  1              2              3                    1000       124
  2              3              5                    1000       124
  3              2              4                    1000       124
  4              4              5                    1000       124
  5              3              4                    1400       124
  7              6              2                    #N/A      #N/A Pump
Node Results:
  ----------------------------------------------------------------------
  Node                Demand      Head  Pressure   Quality
  ID                     LPS         m         m          
  ----------------------------------------------------------------------
  2                     0.00     17.12     17.12      0.00
  3                     2.00     16.16     16.16      0.00
  4                     2.00     16.16     16.16      0.00
  5                     2.00     16.04     16.04      0.00
  6                    -6.00      0.00      0.00      0.00 Reservoir
Link Results:
  ----------------------------------------------------------------------
  Link                  Flow  VelocityUnit Headloss    Status
  ID                     LPS       m/s      m/km
  ----------------------------------------------------------------------
  1                     3.00      0.25      0.96      Open
  2                     1.00      0.08      0.12      Open
  3                     3.00      0.25      0.96      Open
  4                     1.00      0.08      0.12      Open
  5                     0.00      0.00      0.00      Open
  7                     6.00      0.00    -17.12      Open Pump
  \end{verbatim}
     \caption{EPA-NET Summary Report, Simulation \#3}
   \label{fig:epanet3} 
\end{figure}

\begin{figure}[ht!] %  figure placement: here, top, bottom, or page
\centering

\begin{verbatim}
  Page 1                                            10/4/2010 2:29:46 PM
  **********************************************************************
  *                             E P A N E T                            *
  *                     Hydraulic and Water Quality                    *
  *                     Analysis for Pipe Networks                     *
  *                           Version 2.0                              *
  **********************************************************************
  Input File: SIMULATION 3
Link - Node Table:
  ----------------------------------------------------------------------
  Link           Start          End                Length  Diameter
  ID             Node           Node                    m        mm
  ----------------------------------------------------------------------
  1              2              3                    1000       124
  2              3              5                    1000       124
  3              2              4                    1000       124
  4              4              5                    1000       124
  5              3              4                    1400       124
  7              6              2                    #N/A      #N/A Pump
Node Results:
  ----------------------------------------------------------------------
  Node                Demand      Head  Pressure   Quality
  ID                     LPS         m         m          
  ----------------------------------------------------------------------
  2                     0.00     13.52     13.52      0.00
  3                     3.00     11.40     11.40      0.00
  4                     3.00     11.40     11.40      0.00
  5                     3.00     11.15     11.15      0.00
  6                    -9.00      0.00      0.00      0.00 Reservoir
 Link Results:
  ----------------------------------------------------------------------
  Link                  Flow  VelocityUnit Headloss    Status
  ID                     LPS       m/s      m/km
  ----------------------------------------------------------------------
  1                     4.50      0.37      2.12      Open
  2                     1.50      0.12      0.25      Open
  3                     4.50      0.37      2.12      Open
  4                     1.50      0.12      0.25      Open
  5                     0.00      0.00      0.00      Open
  7                     9.00      0.00    -13.52      Open Pump
  \end{verbatim}
     \caption{EPA-NET Summary Report, Simulation \#4}
   \label{fig:epanet4} 
\end{figure}
\clearpage
\clearpage
%%%%%%%%%%%%%%%%%%%%%%%%%%%%%%%%%%%%%%%%%%%

\begin{thebibliography}{}

\bibitem[\protect\citeauthoryear{Swamee and Jain}{Swamee and Jain}{1976}]{jain1976}
Swamee and Jain, A. K., 1976. Explicit equations for pipe-flow problems.  ASCE J. of Hyd. Div., 102(HY5) pp. 657-664 


\end{thebibliography}