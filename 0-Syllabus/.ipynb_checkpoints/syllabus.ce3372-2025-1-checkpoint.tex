\documentclass[12pt]{article}
\usepackage{geometry}                % See geometry.pdf to learn the layout options. There are lots.
\geometry{letterpaper}                   % ... or a4paper or a5paper or ... 
%\geometry{landscape}                % Activate for for rotated page geometry
\usepackage[parfill]{parskip}    % Activate to begin paragraphs with an empty line rather than an indent
\usepackage{daves,fancyhdr,natbib,graphicx,dcolumn,amsmath,lastpage,url}
\usepackage{amsmath,amssymb,epstopdf,longtable}
\usepackage[final]{pdfpages}
\DeclareGraphicsRule{.tif}{png}{.png}{`convert #1 `dirname #1`/`basename #1 .tif`.png}
\pagestyle{fancy}
\lhead{CE 3372 WATER SYSTEMS DESIGN}
\rhead{SPRING 2025}
\lfoot{}
\cfoot{}
\rfoot{Page \thepage\ of \pageref{LastPage}}
\renewcommand\headrulewidth{0pt}

\usepackage{draftwatermark}

% Set the watermark text
\SetWatermarkText{}

% Customize the watermark
\SetWatermarkScale{4}  % Adjust the scale (default is 1)
\SetWatermarkColor[gray]{0.8}  % Set color (gray scale: 0.0 = black, 1.0 = white)
\SetWatermarkAngle{45}  % Set the angle (default is 45 degrees)


\begin{document}
\section*{Syllabus}

%%%%%% BEGIN SYLLABUS COMPONENT %%%%%%%%%%%%%%%%%%%%%%%%%%%%%%%%%%%%%2010-0814TGC%%%%%%%%%%%%%%%%%%%%%%%%%%%%%%%%%%%%%%%%%
\subsection*{{Course Location, Textbook, Instructor Contact Information}}
\begin{tabular}{p{1.5in}p{5.0in}}
Class Meetings:  &    14:00-14:50 , MWF, CE 205 (Section 001) \\
Instructor: & Theodore G. Cleveland, CE Room 203F \\
TA: & none \\
Office Hours:  & T. Cleveland MWF 10:00 - 11:00 \\
~ & T. Cleveland F 13:00-14:00\\
Telephone: & (806)834-5101 \\
%Cell Phone: & (832) 722-4185 (more reliable than office phone) \\
E-mail: & \texttt{theodore.cleveland@ttu.edu}\\
Website: & \texttt{http://54.243.252.9/ce-3372-webroot/} \\
%%%%%%2010-0814TGC%%%%%%%%%%%%%%%%%%%%%%%%%%%%%%%%%%%%%%%%%
Textbook(s) : &  cite{Gupta2017} \\ 
%~ & \cite{Wurbs2002}& \cite{Chin2006} & \cite{mays2011} \\
%~ & \cite{ncees2008} & \cite{rossman2000} & \cite{rossman2009} \\
%~ & \cite{gironas2009} & and more ... & \\
 %\cite{cleveland2013} \\
 %%%%%%2010-0814TGC%%%%%%%%%%%%%%%%%%%%%%%%%%%%%%%%%%%%%%%%%
Copyright : & \textsl{Copyright $\copyright$ 2020 Theodore G. Cleveland, all rights reserved.} \\
\end{tabular}
%%%%%%2010-0814TGC%%%%%%%%%%%%%%%%%%%%%%%%%%%%%%%%%%%%%%%%%
\subsection*{{Catalog Description}}
\begin{quote} \textbf{3372. Water Systems Design (3:3:0).}  Prerequesite: CE 3305\footnote{Mechanics of Fluids}, 3354\footnote{Engineering Hydrology}.  Hydraulic analysis and design of municipal water distribution, stormwater collection, and wastewater collection systems.  Oral and written presentations. (Writing Intensive)
\end{quote}
%%%%%%2010-0814TGC%%%%%%%%%%%%%%%%%%%%%%%%%%%%%%%%%%%%%%%%%
%%%%%%%%%%%%%%%%%%%%%%%%%%%%%%%%%%%%%%%%%%%%%%%%%%%%%%%
%%%%%%%%%%%%%%%%%%%%%%%%%%%%%%%%%%%%%%%%%%%%%%%%%%%%%%%
\subsection*{{Course Objectives}}
The purpose of this class is to study the theory and application of hydraulics, learn how to use predictive tools such as charts and computer programs, and apply these tools to the analysis and design of distribution and collection and pumping systems.  Preparation of professional reports is an equally emphasized objective of this course.
\subsubsection*{{Knowledge, Skills, Abilities (KSA)}}
During this course the student will
\begin{enumerate}
\item Read, synthesize, and communicate ideas presented in current and historical technical literature.
\item Perform flow and transport computation in simplistic topologies using Excel, or \textbf{R} as needed\footnote{This task is principally to develop understanding of how the professional tools function.  Excel and \textbf{R} are professional tool in their own right, but the skill level to use them for engineering is beyond the scope of this class.}.
\item Perform flow and transport computation in realistic geometries using EPA-NET or similar professional tools for water distribution systems.
\item Perform flow and transport computation in realistic geometries using SWMM 5.0 or similar professional tools for stormwater and wastewater collection systems.
\item Size and select engineering materials (pipes, pumps, valves, etc.) for use in the drinking water environment (water distribution), the stormwater environment, and the wastewater environment.
\item Prepare professional reports for the design of a drinking water distribution system, a stormwater collection system, and a wastewater collection system.  

\end{enumerate}

\subsection*{{ABET Program Outcomes}}
This course specifically addresses the outcomes listed below.  Only the primary outcome is assessed as part of the CECE Continuous Improvement Program.

\textbf{Primary Outcomes}
\begin{itemize}
\item (2.)an ability \textbf{to apply engineering design to produce solutions that meet specified needs}
with consideration of public health, safety, and welfare, as well as global, cultural, social,
environmental, and economic factors.\footnote{The boldface portion is the main emphasis in this course}
\item (3.) \textbf{an ability to communicate effectively with a range of audiences}
\item (5.) \textbf{an ability to function effectively on a team whose members together provide
leadership, create a collaborative and inclusive environment, establish goals,
plan tasks, and meet objectives}
\end{itemize}

\textbf{Secondary Outcomes}

\begin{itemize}
\item (7.) An ability to acquire and apply new knowledge as needed, using appropriate learning strategies.
\end{itemize}

\subsubsection*{{About ABET EAC Criteria}}

The outcomes above (primary and secondary) are a subset of those prescribed in ABET EAC General Criteria for Bacculerate Programs \citep{ABET2024}

\quote{\textbf{Criteria 3, Student Outcomes}: The program must have documented student outcomes that support the program educational objectives. Attainment of these outcomes prepares graduates to enter the professional practice of engineering. Student outcomes are outcomes (1) through (7), plus any additional outcomes that may be articulated by the program.}

The primary outcomes (above) are assessed as part of the CECE Continuous Improvement Program that is prescribed in ABET EAC General Criteria for Bacculerate Programs \citep{ABET2024}

\quote{\textbf{Criteria 4, Continuous Improvement}: The program must regularly use appropriate, documented processes for assessing and evaluating the extent to which the student outcomes are being attained. The results of these evaluations must be systematically utilized as input for the program’s continuous improvement actions. Other available information may also be used to assist in the continuous improvement of the program.}



\section*{Course Specific Policies}
Any  policies stated in this section that differ from University Operating Policies are null and void and the University Operating Policies shall be in force.
\subsection*{Disability:}
Texas Tech policy provided as part of syllabus (see last section).

%\textsl{ "Any student who, because of a disability, may require special arrangements in order to meet
%the course requirements should contact the instructor as soon as possible to make any necessary arrangements.
%Students should present appropriate verification from Student Disability Services during the instructors office hours. Please note instructors are not allowed to provide classroom accommodations
%to a student until appropriate verification from Student Disability Services has been provided.
%For additional information, you may contact the Student Disability Services office at 335 West Hall or
%806- 742-2405."}

\subsection*{Religious Holidays:}
\textsl{ "A student who intends to observe a religious holy day (as defined by OP 34.19) should
make that intention known to the instructor prior to the absence in order to receive accommodations
prescribed by OP 34.19."}

\subsection*{Cellphones/Pagers:}
Please set your personal communication devices to silent ring or off during class. 
Do not take calls in class. Disturbance during class time is not acceptable.

\subsection*{Prerequisites:} 
Mastery of material from CE 3305 and CE 3354 or equivalent is expected.

\subsection*{Attendance:} Roll will be taken to determine attendance for class participation.  Please let the instructor know in advance if you must miss a class for a legitimate reason\footnote{Legitimate reasons include: Academically-related extracurricular activities (ASCE, AGU, etc.); Illness with documentation; Federal Family Leave Act Policies; Orders to activate (Military, Peace Officer, Public Health, etc.).  Bring me some kind of documentation for such absences.}. 

\section*{Evaluation Instruments and Grading}
Student performance will be evaluated using attendance as measured by completion of on-line lesson reviews (on the class server), exercises (homework), quizzes (administered on the class server), and examinations  (administered on the class server).   The exams will derive much of their content from the exercises.  %At the end of the semester students are to turn in a portfolio of all graded work.  The portfolio should be comprised of photocopies of exercise materials, article reviews, and exams 1 and 2.  The project report should be submitted under separate cover.  The portfolio should be bound using a binder clip.  The portfolio will not be returned.  

%\subsection*{Article Reviews:} 
%Several article reviews are assigned; part of professional development is reading and interpreting professional literature.  Due dates for these reviews are shown in Table \ref{tab:fall2013schedule}.  These reviews are shown as \texttt{R-\#}.

\subsection*{Exercises:} 
Assignments follow most lectures and are due as listed on the class server.  
%the following class meeting (but consult the schedule below for actual due dates -- changes are announced on the main page of the class website)
\footnote{Legibility, correct method, and correct answer are substantial components of grading criteria.   The grader will not diagnose sources of arithmetic or algebra errors unless the errors are obvious.  Solutions are reviewed in class and posted on the server}.
\begin{enumerate}
\item Every homework assignment is to be accompanied by a descriptive memorandum containing your analysis
of the problem. 
Report materials should be prepared with a word processor. 
Hand computations may be turned in on engineering paper attached as an appendix to the memorandum
%\footnote{Regular binder paper is not acceptable --- use engineering paper.}; important steps in each solution must be shown. 
Legibility is determined by the reader; illegible materials will not be graded.
\item Assignments are to be uploaded to the learning management system (LMS) on the assigned date.\footnote{The use of a LMS (Learning Management System) is an experiment this semester.  The LMS is accessed through the course server.  If the LMS is not working, then assignments are due by midnight on the due date. Late assignments are not accepted.}
\item Due dates are shown in Table \ref{tab:fall2013scheduleA};  Exercises are denoted by \texttt{ES-\#}
\end{enumerate}

\subsection*{Engineering Reports:}  
A project report comprised of various components developed during the course is to be completed.  
The project is introduced early in the semester and is related to the design of a water distribution, stormwater collection, and/or wastewater collection system and accompanying appurtenances.    
There are two "reports" on the schedule: \texttt{RP-1}, and \texttt{RP-2}.  
The reports are to be constructed as a team activity (teams will be selected in the first two weeks of the semester).

\subsection*{Presentations}
An engineering presentation is to be completed and presented to the class for peer evaluation. 
The topic is the same as in \texttt{RP-2}.   The report is due by each team on the day of presentation.

\subsection*{Exams:} Two examinations will be given, they will be of approximately equal difficulty.
\begin{enumerate}
\item Examinations are open notes.
\item Examinations are comprehensive, even though the main focus will be the materials discussed prior to the examination.
\item Full credit for problems will only be given if all computations are documented.
\item Examination dates are shown in Table \ref{tab:fall2013scheduleA}.
\end{enumerate}
\newpage

\section*{Grading:} Final grades are determined based on performance during the semester.  Letter grades will be assigned using University standards.  The \textbf{approximate} weighting of graded material in determining the final grade is as follows:
% Requires the booktabs if the memoir class is not being used
\begin{center}
\begin{table}[h!]
   \begin{tabular}{l l}
Item & Percent of Grade \\
\hline
\hline
Attendance and Participation & ~10\% \\
Project Presentation & ~10\% \\
Project Report-1 & ~10\% \\
Project Report-2 & ~10\% \\
Exercises & 20\% \\
Mid-Term Exam & 20\% \\
Final Exam & 20\% \\
\hline
\end{tabular}
\end{table}
\end{center}
\clearpage
%%%%%%%%%%%%%%%%%%%%%%%%%%%%%%%%%%%%%%%%%%%%%%%%%%%%%%%%%%%%%%%%%%%
\section*{Schedule}
\begin{center}
\begin{table}[ht!]
 \caption{Spring 2025 Course Schedule}
   \begin{tabular}{| p{0.8in} | p{3.5in} | p{2.0in} | } 
\hline
\hline
DATE & TOPIC & READINGS \\
\hline
\hline
\texttt{15 Jan 25} & 1. Introduction & Server Notes   \\
\texttt{17 Jan 25} & 2. Semester Project(s) & Server Notes  \\ %quiz built
\hline
\texttt{20 Jan 25} & MLK Holiday -- no class &  NONE \\
\texttt{22 Jan 25} & 3. Design Guidelines -- Drinking Water Supply  & \cite{Gupta2017} pp. 667-668\\
\texttt{24 Jan 25} & 4. Design Drawings and Maps   & Server Notes \\ 
\hline
\texttt{27 Jan 25} & 5. Demand Estimation -- Population Projections & \cite{Gupta2017} pp. 1-39\\
\texttt{29 Jan 25} & 6. Pipeline Hydraulics -- Review & \cite{Gupta2017} pp. 633-652\\
\texttt{31 Jan 25} & 7. Pumps and Pumping Systems  & \cite{Gupta2017} pp. 672-689 \\
\hline
\texttt{03 Feb 25} & 8. Storage Systems  & \cite{Gupta2017} pp. 668-672\\
\texttt{05 Feb 25} & 9. Pipe Networks -- Background  & \cite{Gupta2017} pp. 656-668 \\
\texttt{07 Feb 25} & 10. Numerical Methods (no pumps)   & \cite{Gupta2017} pp. 664-668 \\
\hline
\texttt{10 Feb 25} & 11. Numerical Methods (with pumps) & \cite{Gupta2017} pp. 664-668 \\
\texttt{12 Feb 25} & Job Fair -- no class & NONE \\
\texttt{14 Feb 25} & 12. Pipeline Transients -- Rigid vs. Elastic & Server Notes\\
\hline
\texttt{17 Feb 25} & 13. Numerical Methods -- Timed valve closure  & Server Notes\\
\texttt{19 Feb 25} & 14. EPANET introduction & Server Notes\\
\texttt{21 Feb 25} & 15. Exam 1 & Blackboard \\
\hline
\texttt{24 Feb 25} & 16. EPANET pumps & Server Notes  \\
\texttt{26 Feb 25} & 17. EPANET demand patterns & Server Notes \\
\texttt{28 Feb 25} & 18. EPANET water quality & Server Notes \\
\hline
\texttt{03 Mar 25} & 19. Project Workshop & Server Notes  \\
\texttt{05 Mar 25} & 20. Design Guidelines -- Storm Water Collection & \cite{Gupta2017} pp. 699-701 \\
\texttt{07 Mar 25} & 21. Stormwater Hydrology -- Review  & \cite{Gupta2017} pp. 711-728 \\
\hline
\texttt{10 Mar 25} & 22. Stormwater Collection --  Inlets & Server Notes\\
\texttt{12 Mar 25} & 23. Stormwater Collection --  Conduit Sizing & \cite{Gupta2017} pp. 729-734\\
\texttt{14 Mar 25} & 24. SWMM introduction & Server Notes \\
\hline
\texttt{24 Mar 25} & 25. SWMM hydrology & Server Notes  \\
\texttt{26 Mar 25} & 26. SWMM conduits & Server Notes \\ 
\texttt{28 Mar 25} & 27. SWMM storage and pumps & Server Notes \\ 
\hline
   \end{tabular}
   \label{tab:sp2025scheduleA}
\end{table}
\clearpage
%%%%%%%%%%%%%%%%%%%%%%%%%%%%%%%%%%%%%%%%%%%%%%%%%%%%%%%
\begin{table}[ht!]
 \caption{Spring 2025 Course Schedule - Continued}
   \begin{tabular}{| p{0.8in} | p{3.5in} | p{2.0in} |} 
\hline
\hline
DATE & TOPIC & READINGS \\
\hline
\hline
\texttt{31 Mar 25} & 28. Detention Ponds & Server Notes \\
\texttt{02 Apr 25} & 29. Green-Infrastructure & Server Notes \\
\texttt{04 Apr 25} & 30. Exam 2 & Blackboard \\
\hline
\texttt{07 Apr 25} & 31. SWMM water quality & Lesson-22 \\
\texttt{09 Apr 25} & 32. Design Guidelines -- Sanitary Sewer Systems & \cite{Gupta2017} pp. 702-711 \\
\texttt{11 Apr 25} & 33. Project Workshop & Blackboard \\
\hline
\texttt{14 Apr 25} & 34. Waste Water Collection & \cite{Gupta2017} pp. 702-711 \\
\texttt{16 Apr 25} & 35. Inflow and Infiltration & Server Notes \\
\texttt{18 Apr 25} & 36. Outfall Considerations  & Server Notes \\
\hline
\texttt{21 Apr 25} & Holiday -- no class & NONE \\
\texttt{23 Apr 25} & 37. Project Workshop & Server Notes \\
\texttt{25 Apr 25} & 38. Project Workshop & Server Notes  \\
\hline
\texttt{28 Apr 25} & 39. Project Presentations Teams 1-3 & Server Notes   \\
\texttt{30 Apr 25} & 40. Project Presentations Teams 4-6 & Server Notes   \\
\texttt{02 May 25} & 41. Project Presentations Teams 7-9 & Server Notes   \\
\hline
\texttt{05 May 25} & 42. Project Presentations Teams 10-12 & Server Notes   \\
\hline
\hline
\texttt{xx May 25} & Exam 3 & Blackboard \\
\hline
\hline
   \end{tabular}
   \label{tab:sp2025scheduleB}
\end{table}
\end{center}
\clearpage
%%%%%%%%%%%%%%%%%%%%%%%%%%%%%%%%%%%%%%%%%
\begin{thebibliography}{}
\bibitem[Gupta(2017)]{Gupta2017}
Gupta, R.S. 2017. Hydrology and Hydraulic Systems 4th ed. Waveland Press, Inc. ISBN 978-1-4786-3091-3 888p.

\bibitem[ABET(2024)]{ABET2024}
ABET (2024) Criteria for Accrediting ENGINEERING PROGRAMS. 
ABET, 415 North Charles St., Baltimore, MD 21201 
[FREE DOWNLOAD] @ \url{https://www.abet.org/wp-content/uploads/2023/03/23-24-EAC-Criteria_FINAL2.pdf}

\bibitem[\protect\citeauthoryear{Chin}{Chin}{2006}]{Chin2006}
Chin, D. (2006). 
\newblock{\em Water Resources Engineering, 2 ed.}
\newblock Prentice Hall, Inc.

\bibitem[\protect\citeauthoryear{Mays}{Mays}{2011}]{mays2011}
Mays, L.~W. (2011).
\newblock {\em Water-Resources Engineering}.
\newblock Wiley.

\bibitem[\protect\citeauthoryear{Wurbs}{Wurbs}{2002}]{Wurbs2002}
Wurbs, R.A., and James, W. P. (2002).
\newblock{\em Water Resources Engineering}
\newblock Prentice Hall; pp.130-156; and 156-198. 

\bibitem[\protect\citeauthoryear{NCEES}{NCEES}{2008}]{ncees2008}
NCEES (2008).
\newblock {\em Fundamentals of Engineering Supplied Reference Handbook\/} (8th
  ed.).
\newblock 280 Seneca Creek Road, Clemson, SC 29631: National Council of
  Examiners for Engineering and Surveying {ISBN 978-1-932613-37-7}.

\bibitem[\protect\citeauthoryear{Roberson}{Roberson}{1988}]{roberson1988}
Roberson, J.~A., Cassidy, J. ~J., and Chaudry, M.~H.  (1988).
\newblock {\em Hydraulic Engineering}.
\newblock Houghton Mifflin.

\bibitem[\protect\citeauthoryear{Rossman}{Rossman}{2000}]{rossman2000}
Rossman, L. (2000).
\newblock {EPANET 2} users manual.
\newblock Technical Report EPA/600/R-00/057, U.S. Environmental Protection
  Agency, National Risk Management Research Laboratory Cincinnati, OH 45268.
  
\bibitem[\protect\citeauthoryear{Rossman}{Rossman}{2009}]{rossman2009}
Rossman, L. (2009).
\newblock {Storm Water Management Model} user's manual version 5.0.
\newblock Technical Report EPA/600/R-05/040, U.S. Environmental Protection
  Agency, National Risk Management Research Laboratory Cincinnati, OH 45268.

%==========STANDARD COURSE POLICY MATERIALS, SHOULD NOT CHANGE OFTEN=======

\section*{University Policies}
Policies stated in this section override any policies in the course specific policies section above.
~\\~\\
These University Operating Policies are provided as directed and cover institutionally required information including: 
ADA Statement, Academic Integrity, Religious Holy Day Statement.
\\~\\
Additionally the institutionally suggested statements are also included in the syllabus.  
These statements cover topics related to discrimination, civility, and diversity.  

~\\The University ADA Policy is presented verbatim -- students requesting accommodations must do so using the procedures defined in the policy.

\includepdf[pages=-]{./StatementsRequiredandRecommendedforCourseSyllabi_2017.pdf}
\includepdf[pages=-]{./OP10-08.pdf}
%\bibitem[\protect\citeauthoryear{Gironas, Roesner, and Davis}{Gironas
% et~al.}{2009}]{gironas2009}
%Gironas, J., L.~A. Roesner, and J.~Davis (2009).
%\newblock {Storm Water Management Model} applications manual.
%\newblock Technical Report EPA/600/R-09/077, U.S. Environmental Protection
%  Agency, National Risk Management Research Laboratory Cincinnati, OH 45268.

\end{thebibliography}



\end{document}  