\documentclass[12pt]{article}
\usepackage{geometry}                % See geometry.pdf to learn the layout options. There are lots.
\geometry{letterpaper}                   % ... or a4paper or a5paper or ... 
%\geometry{landscape}                % Activate for for rotated page geometry
\usepackage[parfill]{parskip}    % Activate to begin paragraphs with an empty line rather than an indent
\usepackage{daves,fancyhdr,natbib,graphicx,dcolumn,amsmath,lastpage,url}
\usepackage{amsmath,amssymb,epstopdf,longtable}
\usepackage{paralist} 
\usepackage[final]{pdfpages}
\DeclareGraphicsRule{.tif}{png}{.png}{`convert #1 `dirname #1`/`basename #1 .tif`.png}
\pagestyle{fancy}
\lhead{CE 3372 -- Water Systems Design}
\rhead{SPRING 2025}
\lfoot{EXERCISE 1}
\cfoot{}
\rfoot{Page \thepage\ of \pageref{LastPage}}
\renewcommand\headrulewidth{0pt}

\begin{document}
\begin{center}
{\textbf{{ CE 3372 -- Water Systems Design} \\ {Exercise 1}}}
\end{center}
\section*{{Purpose}} 
\begin{itemize}
\item Access to the class website and upload completed exercises as .PDF files.
\item Use web resources and supplied readings to self-teach about water systems.
\end{itemize}
\section*{Exercises}

\begin{enumerate}
%\item{{\textbf{Relevant Water Systems Web Resources}}}~\\~\\
%Visit the \texttt{URLs} in Table \ref{tab:url-list} below, and complete the table entries.  
%In this exercise, governmental means the entity has taxing authority and elected officials.
%Use the following ``codes'' for the type of organization based on your web visit.
%\begin{enumerate}[a)]
%\item GOV:$<$level$>$ == A municipality or city is City; A county is County; A state entity is State; and a Federal entity is Federal.  
%For example, a visit to a city level website is coded as: GOV:City.
%\item NGO == A non-governmental entity.  A professional organization is one example, Amnesty International is another example.
%\end{enumerate}
%% Requires the booktabs if the memoir class is not being used
%\begin{table}[htbp]
%   \centering
%   \caption{Water Relevant Websites in the United States}
%   \begin{tabular}{p{2.5in}p{3.5in}p{0.5in}} % Column formatting, @{} suppresses leading/trailing space
%   ~ & ~ & ~ \\
%\hline \hline
%\texttt{URL} & Organization Name & Type \\
%\hline
%http://www.asce.org & American Society of Civil Engineers & NGO \\
%%http://www.water.ca.gov && \\
%https://www.ci.lubbock.tx.us && \\
%https://www.tceq.texas.gov && \\
%%http://www.edwardscountytexas.us && \\
%https://www.hcfcd.org && \\
%http://www.dep.state.fl.us && \\
%%http://www.usace.army.mil && \\
%https://www.nrcs.usda.gov && \\
%https://www.usgs.gov && \\
%%https://tx.usgs.gov && \\
%https://www.epa.gov && \\
%https://www.awwa.org && \\
%%http://www.wef.org && \\
%http://www.ngwa.org && \\
%https://www.fhwa.dot.gov && \\
%%http://www.iwra.org && \\
%http://www.asce.ewrinstitute.org && \\
%\hline
%\hline
%   \end{tabular}
%   \label{tab:url-list}
%\end{table}
%\clearpage
%%%%%%%%%%%%%%%%%%%%%%%%%%
%%%%%%%%%%%%%%%%%%%%%%%%%%
\item{{\textbf{Water Use Descriptions}}}~\\~\\
Consult the readings \citep{Chin2006,mays2011,Wurbs2002} for Lecture 1 and use your findings to complete the water use descriptions in Table \ref{tab:water-use}, below.
\begin{table}[htbp]
   \centering
   \caption{Water Use Categories}
   \begin{tabular}{p{1.7in}p{4.3in}} % Column formatting, @{} suppresses leading/trailing space
   ~ & ~ \\
\hline \hline
Water Use Category & Description \\
\hline
Domestic Use & Water for household needs: drinking, food preparation, personal hygiene, washing clothes and dishes, flushing toilets, and watering lawns and gardens (also called residential use) \\
   ~ & ~ \\
Commercial Use & ~\\
   ~ & ~ \\
Irrigation Use & ~\\
   ~ & ~ \\
Industrial Use & ~\\
   ~ & ~ \\
Livestock Use & ~\\
   ~ & ~ \\
%Mining Use & ~\\
%   ~ & ~ \\
%Public Use & ~\\
%   ~ & ~ \\
%Rural Use & ~\\
%   ~ & ~ \\
%Thermoelectric Use & ~\\
%   ~ & ~ \\
Hydroelectric Use & ~\\
   ~ & ~ \\
%Environmental Use & ~\\
%   ~ & ~ \\
Recreational Use & ~\\
   ~ & ~ \\
Navigation (Inland) Use & ~\\
   ~ & ~ \\
\hline
\hline
   \end{tabular}
   \label{tab:water-use}
\end{table}
\clearpage
\item{{\textbf{Climate Conditions -- 1}}}~\\~\\
Consult \cite{Chin2006} and use your findings to complete the climate range table, Table \ref{tab:climate} below in U.S. Customary units as indicated.

\begin{table}[htbp]
   \centering
   \caption{Climate Description, Precipitation, and Evapotranspiration}
   \begin{tabular}{p{1.5in}p{2.25in}p{2.25in}} % Column formatting, @{} suppresses leading/trailing space
   ~ & ~ & ~\\
\hline \hline
Climate Description & Mean Annual Precip. (inches) & Mean Annual Evap. (inches) \\
\hline
   ~ & ~ & ~\\
   Superarid & $<~$4 & $<~$118 \\
      ~ & ~ \\
   Hyperarid & & \\
   ~ & ~ \\   
   Arid && \\
   ~ & ~ \\   
   Semiarid && \\
   ~ & ~ \\   
   Subhumid && \\
   ~ & ~ \\   
   Humid && \\
   ~ & ~ \\   
   Hyperhumid && \\
   ~ & ~ \\   
   Superhumid && \\
   ~ & ~ \\   
\hline
\hline
   \end{tabular}
   \label{tab:climate}
\end{table}
\clearpage
\item{{\textbf{Climate Conditions -- 2}}}~\\~\\
Use the internet and locate values of mean annual precipitation and mean annual evapotranspiration for Lubbock, Texas; Albuquerque, New Mexico, and Houston, Texas.   Using Table \ref{tab:climate} you constructed above, classify the climates of these three locales.
\end{enumerate}




%%%%%%%%%%%%%%%


\begin{thebibliography}{\small}

\bibitem[\protect\citeauthoryear{Chin}{Chin}{2006}]{Chin2006}
Chin, D. (2006). 
\newblock{\em Water Resources Engineering, 2 ed.}
\newblock Prentice Hall, Inc. {pp. 1--8} 
\newblock \url{http://www.rtfmps.com/documents/university-courses/ce-3372/3-Readings/Chin1-8/}

\bibitem[\protect\citeauthoryear{Mays}{Mays}{2011}]{mays2011}
Mays, L.~W. (2011).
\newblock {\em Water-Resources Engineering}.
\newblock Wiley. {pp. 1--11}
\newblock \url{http://www.rtfmps.com/documents/university-courses/ce-3372/3-Readings/Mays1-11/}

\bibitem[\protect\citeauthoryear{Wurbs}{Wurbs}{2002}]{Wurbs2002}
Wurbs, R.A., and James, W. P. (2002).
\newblock{\em Water Resources Engineering}
\newblock Prentice Hall. {pp. 1--33}
\newblock \url{http://www.rtfmps.com/documents/university-courses/ce-3372/3-Readings/Wurbs1-33/}

\end{thebibliography}
\end{document}  