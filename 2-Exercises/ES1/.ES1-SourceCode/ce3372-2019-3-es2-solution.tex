\documentclass[12pt]{article}
\usepackage{geometry}                % See geometry.pdf to learn the layout options. There are lots.
\geometry{letterpaper}                   % ... or a4paper or a5paper or ... 
%\geometry{landscape}                % Activate for for rotated page geometry
\usepackage[parfill]{parskip}    % Activate to begin paragraphs with an empty line rather than an indent
\usepackage{daves,fancyhdr,natbib,graphicx,dcolumn,amsmath,lastpage,url}
\usepackage{amsmath,amssymb,epstopdf,longtable}
\usepackage{paralist} 
\usepackage[final]{pdfpages}
\DeclareGraphicsRule{.tif}{png}{.png}{`convert #1 `dirname #1`/`basename #1 .tif`.png}
\pagestyle{fancy}
%%%%%%%%%%%%%%%%%%%%%%%%%%%%%%
%%%%%%%%%%%%%%%%%%%%%%%%%%%%%%
\lhead{CE 3372 -- Water Systems Design}
\rhead{FALL 2019}
\lfoot{EXERCISE 2}
\cfoot{}
\rfoot{Page \thepage\ of \pageref{LastPage}}
\renewcommand\headrulewidth{0pt}
\newcommand\tab[1][1cm]{\hspace*{#1}}
%%%%%%%%%%%%%%%%%%%%%%%%%%%%%%%
%%%%%%%%%%%%%%%%%%%%%%%%%%%%%%%
\begin{document}
\begin{center}
\textbf{MEMORANDUM}
%{\textbf{{ CE 3372 -- Water Systems Design} \\ {Exercise Set 2}}}
\end{center}
\begingroup
\begin{tabular}{p{1in} p{5in}}
To: & P. N. Guin \\ ~\\
From: & P. Olar Bear \\ ~\\
Date: & 04JAN2024 \\ ~\\
Subject: & CE 3372 -- Water Systems Design, Exercise Set 2. ~\\


\end{tabular}
\endgroup
\section*{\small{Purpose}}  
This memorandum is a brief summary and review of the article ``History Lesson'' regarding the design and construction of the Los Angeles Aqueduct.

\section*{\small{Discussion}}
The article presents a brief history of the Los Angeles Aquaduct as built in the early 20-th century.  The system captures water from snowmelt in the Owens Valley on the Eastern Sierra Nevada range and transports that water to Los Angeles.   
The system is gravity flow (in that the Owens Valley is at considerably greater altitude than Los Angeles), and contains several electricity generating stations.  Most of the system is open channel in either open trapezoidal channels or large-bore tunnels dug through rock.  Some portions are pressurized pipes that cross valleys as inverted siphons.

The project itself involved construction of roads, rail extensions, concrete plants, and mining operations.   
It was, for its time, a huge engineering undertaking involving nearly all aspects of civil engineering, including land acquisition by both purchase and the legal concept of imminent domain.  

The techniques in engineering and construction were later used in the construction of large dams in the Western USA, including Boulder (Hoover) Dam, and Grand Coulee Dam, which arguably contributed to the economic sucess of the USA since the 1930s.
%The later dam generated the electricity to produce aluminum which was assembled by the Boeing Aircraft Corporation into aircraft in support of the war effort against the Axis powers in WWII.   

The aqueduct is a living example that encompasses most existing water transmission technologies; except for pumping.  The aqueduct has pressurized pipeline portions, turbines (energy recovery), open channel portions, and storage components.   When compared to a modern gas pipeline the principle differences are the working fluid, and the requirement in a gas pipeline of compressor (pumping) stations to maintaining the driving force to transmit the product.  Otherwise in terms of scale, complexity, and importance the two are equivalent.



\end{document}  