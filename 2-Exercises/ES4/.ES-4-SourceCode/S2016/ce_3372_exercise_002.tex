\documentclass[12pt]{article}
\usepackage{geometry}                % See geometry.pdf to learn the layout options. There are lots.
\geometry{letterpaper}                   % ... or a4paper or a5paper or ... 
%\geometry{landscape}                % Activate for for rotated page geometry
\usepackage[parfill]{parskip}    % Activate to begin paragraphs with an empty line rather than an indent
\usepackage{daves,fancyhdr,natbib,graphicx,dcolumn,amsmath,lastpage,url}
\usepackage{amsmath,amssymb,epstopdf,longtable}
\usepackage{paralist} 
\DeclareGraphicsRule{.tif}{png}{.png}{`convert #1 `dirname #1`/`basename #1 .tif`.png}
\pagestyle{fancy}
\lhead{CE 3372 -- Water Systems Design}
%\rhead{FALL 2010}
%\rhead{SPRING 2011}
%\rhead{FALL 2011}
%\rhead{SPRING 2012}
%\rhead{FALL 2012}
\rhead{FALL 2015}
%\lfoot{EXERCISE 1 -- REVISION 1}
%\lfoot{EXERCISE 1 -- REVISION 2}
%\lfoot{EXERCISE 1 -- REVISION 3}
%\lfoot{EXERCISE 1 -- DUE 26 JAN 2012}
%\lfoot{EXERCISE 1 -- DUE 4 SEP 2012}
\lfoot{EXERCISE 2}
\cfoot{}
\rfoot{Page \thepage\ of \pageref{LastPage}}
\renewcommand\headrulewidth{0pt}



\begin{document}
\begin{center}
{\textbf{{ CE 3372 -- Water Systems Design} \\ {Exercise Set 2}}}
\end{center}
\begingroup
\begin{tabular}{p{1in} p{5in}}
Purpose: & Reinforce use of authoritative documents as sources of design criteria. \\
~             & Reinforce reading and interpreting plan and profile drawings \\
~             & Review of closed conduit hydraulics and use of energy equation for systems analysis \\
\end{tabular}
\endgroup
\section*{\small{Exercises}}
\begin{enumerate}
%%%%%%%%%%%%%%%% TEXAS ADMINISTRATIVE CODE %%%%%%%%%%%%%%%%%%%%%%%%%%%%%%%%%%%%%%%%%%%%%%%%%%%%%%
\item Locate TCEQ Rules and Regulations for Public Water Systems (RG-195).    Use the document to answer the following questions
\begin{enumerate}
\item  What is the minimum required pressure in a distribution system under normal operating conditions?
\item What is the minimum required pressure in a distribution system under combined fire and drinking water operating conditions?
\item What is the minimum distance that waterlines can be located relative to septic tank drainfields?
\item What is the minimum required residual of free chlorine in milligram per liter?
\item What is the minimum required residual of chloramine residual for systems that feed ammonia?
\end{enumerate}
\item Locate Chapter 217 ``Design Criteria for Domestic Wastewater Systems'' of the Texas Administrative Code.   Download and print Subchapter C ``Conventional Collection Systems''.   Use the subchapter to answer the following questions.
\begin{enumerate}[a)]
\item What four components of flow calculations are to be included in a gravity collection system?
\item What time frame are these flow calculations to be considered?
\item Collection system pipes may be installed in the same trench as a pressurized water supply pipe.  True or false?
%\item Collection system surcharge is allowed when actual flow equals expected peak design flow.  True or false?
\item What is the minimum structural design life for wastewater collection facilities?
\item What is the minimum allowable gravity collection pipe diameter, in feet?
\item What is the minimum allowed flow velocity when a collection system component is flowing full?
\item What is the maximum allowed flow velocity when a collection system component is flowing full?
\item What is the head loss formula is used in the gravity collection portion of the subchapter?
\item What is the head loss formula used in the lift station (pumping) portion of the subchapter?
\item Collection system warning labels must be provided in what languages ?
%\item What ASTM, ASCE, and ANSI standards are referenced in the subchapter with regards to trench backfill?
\end{enumerate}
%%%%%%%%%%%%%%%%% CITY OF HOUSTON DESIGN MANUAL %%%%%%%%%%%%%%%%%%%%%%%%%%%%%%%%%%%%%%%%%%%%%%%%%
\item Using the City of Houston Infrastructure Design Manual on the class website, summarize backfill requirements as pertaining to wastewater collection systems. 
%%%%%%%%%%%%%%%%% PLAN AND PROFILE %%%%%%%%%%%%%%%%%%%%%%%%%%%%%%%%%%%%%%%%%%%%%%%%%%%%%%%%%%
%\item Using the City of Rocksprings Storm Sewer Plan and Profile on the class website determine:
%\begin{enumerate}
%\item What is the station of inlet 8?
%\item What is the offset of inlet 8?
%\item What is the elevation that connects IL8 and IL8A to the main line?
%\end{enumerate}
%\item	What is the ADA requirement for ramp slopes? Be sure to cite your source.   
%%%%%%%%%%%%%%%%% CLOSED CONDUIT HYDRAULICS %%%%%%%%%%%%%%%%%%%%%%%%%%%%%%%%%%%%%%%%%%%%%%
\item A 5-foot diameter, enamel coated, steel pipe carries 60$^o$F water at a discharge of 295 cubic-feet per second (cfs).  Estimate the head loss in a 10,000 foot length of this pipe using the Hazen-Williams head loss model.

%\item Estimate the diameter of a cast-iron pipe needed to carry 60$^o$F water at a discharge of 10 cubic-feet per second (cfs) between two reservoirs 2 miles apart.  The elevation difference between the water surfaces in the two reservoirs is 20 feet.

%\item Points A and B are 3 miles apart along a 24-inch ductile iron pipe carrying 50$^o$F water.   Point A is 30 feet higher in elevation than point B.  The pressure at point B is 20 pounds per square-inch (psi) greater than the pressure at point A.  Determine the direction and magnitude of flow (i.e. discharge and direction).

\item  Water is pumped from a supply reservoir to a ductile iron water-transmission line as shown in Figure \ref{fig:p219}.   The high elevation of the line is at point A, 1 kilometer downstream of the pump station, and the low elevation is at point B, 1 kilometer downstream of point A.  If the discharge in the pipeline is 1 cubic-meter per second (cms), the diameter of the pipe is 750 millimeters (mm) and the pressure at point A is 350 kilopascals (kPa).  Determine

\begin{enumerate}[(a)]
\item The added head supplied by the pump station.
\item The water pressure at B.
\item The mechanical power supplied by the pump.
\end{enumerate}

\begin{figure}[htbp] %  figure placement: here, top, bottom, or page
   \centering
   \includegraphics[width=4in]{p219.pdf} 
   \caption{Reservoir-Pump-Transmission System Schematic}
   \label{fig:p219}
\end{figure}

%%%%%%%%%%%%%%%%%%%%%%%%%%%%%%%%%%%%%%%%%%%%%%%%%%%%%%%%%%%%%%%%%%%%%%%%%%%%%%
\end{enumerate}
\end{document}  