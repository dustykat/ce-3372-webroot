\documentclass[12pt]{article}
\usepackage{geometry}                % See geometry.pdf to learn the layout options. There are lots.
\geometry{letterpaper}                   % ... or a4paper or a5paper or ... 
%\geometry{landscape}                % Activate for for rotated page geometry
\usepackage[parfill]{parskip}    % Activate to begin paragraphs with an empty line rather than an indent
\usepackage{daves,fancyhdr,natbib,graphicx,dcolumn,amsmath,lastpage,url}
\usepackage{amsmath,amssymb,epstopdf,longtable}
\usepackage{paralist} 
\DeclareGraphicsRule{.tif}{png}{.png}{`convert #1 `dirname #1`/`basename #1 .tif`.png}
\pagestyle{fancy}
\lhead{CE 3372 -- Water Systems Design}
\rhead{SPRING 2026}
%\rhead{FALL 2016}
%\rhead{SPRING 2016}
%\rhead{FALL 2011}
%\rhead{SPRING 2012}
%\rhead{FALL 2012}
%\rhead{FALL 2015}
%\rhead{FALL 2010}
%\lfoot{EXERCISE 1 -- REVISION 1}
%\lfoot{EXERCISE 1 -- REVISION 2}
%\lfoot{EXERCISE 1 -- REVISION 3}
%\lfoot{EXERCISE 1 -- DUE 26 JAN 2012}
%\lfoot{EXERCISE 1 -- DUE 4 SEP 2012}
\lfoot{EXERCISE 4}
\cfoot{}
\rfoot{Page \thepage\ of \pageref{LastPage}}
\renewcommand\headrulewidth{0pt}
\newcommand\tab[1][1cm]{\hspace*{#1}}


\begin{document}
\begin{center}
{\textbf{{ CE 3372 -- Water Systems Design} \\ {Exercise Set 4}}}
\end{center}
\begingroup
\begin{tabular}{p{1in} p{5in}}
Purpose: & Review and application of head loss models used in water transmission and distribution system analysis\\


\end{tabular}
\endgroup
\section*{\small{Exercises}}
\begin{enumerate}
%%%%%%%%%%%%%%%% TEXAS ADMINISTRATIVE CODE %%%%%%%%%%%%%%%%%%%%%%%%%%%%%%%%%%%%%%%%%%%%%%%%%%%%%%

\item Equation \ref{eqn:hazen-williams} is the  Hazen-Williams discharge formula in US Customary Units. 
\begin{equation}
Q = 1.318 C_h A R^{0.63} S^{0.54}
\label{eqn:hazen-williams}
\end{equation}
where;\\
\tab $Q$ is the discharge in $ft^3/sec$;\\
\tab $A$ is the cross section area of pipe in $ft^2$ ($A = \frac{\pi D^2}{4}$; $D$ is the pipe diameter.);\\
\tab $C_h$ is the Hazen-Williams friction coefficient (depends on pipe roughness);\\
\tab $R$ is the hydraulic radius in $ft$; and \\
\tab $S$ is the slope of the energy grade line ($\frac{h_f}{L}$); $L$ is the length of pipe.
\begin{enumerate}[(a)]
%\item Rearrange the equation in terms of head loss ($h_f = \dots$). 
\item Look up the Hazen-Williams loss coefficient ($C_h$) for enamel coated, steel pipe and cite your data source.
\item Estimate the head loss in a 10,000 foot length of 5-foot diameter, enamel coated steel pipe that carries carries 60$^o$F water at a discharge of 295 cubic-feet per second (cfs), using the Hazen-Williams head loss model.
\end{enumerate}
\clearpage
\item Equation \ref{eqn:hazen-williamsSI} is the  Hazen-Williams discharge formula in SI Units. 
\begin{equation}
Q = 0.849 C_h A R^{0.63} S^{0.54}
\label{eqn:hazen-williamsSI}
\end{equation}
where;\\
\tab $Q$ is the discharge in $m^3/sec$;\\
\tab $A$ is the cross section area of pipe in $m^2$ ($A = \frac{\pi D^2}{4}$; $D$ is the pipe diameter.);\\
\tab $C_h$ is the Hazen-Williams friction coefficient (depends on pipe roughness);\\
\tab $R$ is the hydraulic radius in $m$; and \\
\tab $S$ is the slope of the energy grade line ($\frac{h_f}{L}$); $L$ is the length of pipe.
\begin{enumerate}[(a)]
%\item Rearrange the equation in terms of head loss ($h_f = \dots$). 
\item Look up the Hazen-Williams loss coefficient ($C_h$) for Acrylonite Butadiene Styrene (ABS) pipe and cite your data source.
\item Estimate the head loss in a 3,050 meter length of 1.5-meter diameter, ABS pipe that carries carries 20$^o$C water at a discharge of 8.35 cubic-meters per second (cms), using the Hazen-Williams head loss model.
\end{enumerate}
\clearpage
%%%%%%%%%%%%%%%%%%%%%%%%%%%%%%%%%%%%%%%%%%%%%%%%%%%%%%%%%%%%%%%%%%%%%%%%%%%%%%%%%%%%%%%%%%%%%%%%%%%%%
\item 
Equation \ref{eqn:flow-jain} is an explicit formula (based on the Darcy-Weisbach head loss model and the Colebrook-White frictional loss equation)for estimating discharge from head loss and material properties \citep{jain1976}.
\begin{equation}
Q=-2.22D^{5/2} \times \sqrt{gh_f/L}\times[log_{10} (\frac{k_s}{3.7D} + \frac{1.78\nu}{D^{3/2}\sqrt{gh_f/L}} )]
\label{eqn:flow-jain}
\end{equation}

where;\\~\\
\tab $Q$ is the discharge in $L^3/T$;\\
\tab $D$ is the pipe diameter; \\
\tab $h_f$ is the head loss in the pipe; \\
\tab $g$ is the gravitational acceleration constant; \\
\tab $L$ is the length of pipe; \\
\tab $k_s$ is the pipe roughness height; \\
\tab $\nu$ is the viscosity of liquid in the pipe; \\ 


\begin{enumerate}[(a)]
\item Find the viscosity for water at 50$^o$F.  Cite the source of your value. 
\item Find the sand roughness height of ductile iron pipe.  Cite the source of your value. 
\item How deep is a column of water if the pressure at the bottom of the column is 20 psi?
\item Estimate the discharge in the 3 mile long, 24-inch diameter, ductile iron pipeline connecting points A and B depicted in Figure \ref{fig:PipePressureProblem}.  Point A is 30 feet higher in elevation than point B.  The pressure at point B is 20 pounds per square-inch (psi) greater than the pressure at point A.
\end{enumerate}

\begin{figure}[htbp] %  figure placement: here, top, bottom, or page
   \centering
   \includegraphics[width=4in]{PipePressureProblem.jpg} 
   \caption{Pipeline Schematic}
   \label{fig:PipePressureProblem}
\end{figure}
\clearpage
%%%%%%%%%%%%%%%%%%%%%%%%%%%%%%%%%%%%%%%%%%%%%%%%%%%%%%%%%%%%%%%%%%%%%%%%%%%%%%%%%%%%%%%%%%%%%%%%%%%%%%%
\item 
Equation \ref{eqn:diameter-jain} is a formula to estimate the required pipe diameter for a particular discharge, head loss, and roughness \citep{jain1976}.
\begin{equation}
D=0.66[k_s^{1.25}\times(\frac{LQ^2}{gh_f})^{4.75}+\nu Q^{9.4}\times(\frac{L}{gh_f})^{5.2}]^{0.04}
\label{eqn:diameter-jain}
\end{equation}

where;\\~\\
\tab $D$ is the pipe diameter; \\
\tab $k_s$ is the pipe roughness height; \\
\tab $L$ is the length of pipe; \\
\tab $g$ is the gravitational acceleration constant; \\
\tab $Q$ is the discharge in $L^3/T$;\\
\tab $h_f$ is the head loss in the pipe; \\
\tab $\nu$ is the viscosity of liquid in the pipe; \\ 

\begin{enumerate}[(a)]
\item Find the viscosity for water at 60$^o$F.  Cite the source of your value.
\item Find the sand roughness height of cast-iron pipe.  Cite the source of your value. 
\item Estimate the diameter of a cast-iron pipe needed to carry 60$^o$F water at a discharge of 10 cubic-feet per second (CFS) between two reservoirs 2 miles apart with an elevation difference between the water surfaces in the two reservoirs of 20 feet as depicted in Figure \ref{fig:PipeLine2Reservoirs}.
\end{enumerate}

\begin{figure}[htbp] %  figure placement: here, top, bottom, or page
   \centering
   \includegraphics[width=4in]{PipeLine2Reservoirs.jpg} 
   \caption{Pipeline connecting two reservoirs}
   \label{fig:PipeLine2Reservoirs}
\end{figure}
\clearpage
%%%%%%%%%%%%%%%%%%
% Tehachapi Pipelines YOYO %%%
%%%%%%%%%%%%%%%%%%
\item Figure \ref{fig:parallelpipes} is an aerial image of a parallel pipeline system in California.   

\begin{figure}[htbp] %  figure placement: here, top, bottom, or page
   \centering
   \includegraphics[height=5in]{parallelpipes.jpg} 
   \caption{Parallel Pipeline System}
   \label{fig:parallelpipes}
\end{figure}

The left pipeline is a 96-inch diameter steel pipe, whereas the right pipeline is a 108-inch diameter steel pipe.  
Water at 50$^o$F has kinematic viscosity of $1.45\times10^{-5}~ft^2/s$.   
The sand roughness of ductile iron is $1.64\times10^{-4}~ft$.   
If the head difference for the one-mile long pipelines between the thrust blocks is 120 feet, determine the discharge in each pipe in cubic-feet-per-second.


%\item A water supply system draws from a river at an elevation of 800-feet and delivers the water to a storage reservoir at elevation 820-feet.  The supply pipeline is a 1000-foot long, 10-inch diameter, cast iron pipe.  A single pump with the pump characteristic curve in Figure \ref{fig:PumpCurve} is used to fill the reservoir.
%
%Determine:
%\begin{enumerate}[a)]
%\item Sketch the system described in the problem statement.
%\item Inlet and outlet minor loss coefficients, cite your source of minor loss coefficients.
%\item The roughness ratio for use in the Moody chart, cite your source of roughness height.
%\item The energy equation for the system.
%\item The system loss for a discharge of 1200, 1600, 2000, 2400, and 2800 gallons-per-minute.  Show the calculation of Reynolds number for the different flow rates.  Show the the friction factors on the attached Moody chart (Figure \ref{fig:moody}).
%\item The operating discharge for the system using the supplied pump curve.
%\item The electric power supplied to the pump to lift the water at the operating point.\footnote{Assume the efficiency on the pump curve is representative of the wire-to-water efficiency.}
%\end{enumerate}
%
%\begin{figure}[h!] %  figure placement: here, top, bottom, or page
%\centering
%   \includegraphics[width=4.5in]{PumpCurve.jpg}
%   \caption{Pump characteristic curve}
%   \label{fig:PumpCurve} 
%\end{figure}
%
%\clearpage
%\begin{figure}[h!] %  figure placement: here, top, bottom, or page
%\centering
%   \includegraphics[width=5.5in]{Moody1.jpg}
%   \caption{Moody-Stanton Chart}
%   \label{fig:moody} 
%\end{figure}
%\clearpage

%%%%%%%%%%%%%%%%%%%%%%%%%%%%%%%%%%%%%%%%%%%%%%%%%%%%%%%%%%%%%%%%%%%%%%%%%%%%%%%%%%%%%%%%%%%%%%%%

%\item  Water is pumped from a supply reservoir to a ductile iron water-transmission line as shown in Figure \ref{fig:p219}.   The high elevation of the line is at point A, 1 kilometer downstream of the pump station, and the low elevation is at point B, 1 kilometer downstream of point A.  If the discharge in the pipeline is 1 cubic-meter per second (cms), the diameter of the pipe is 750 millimeters (mm) and the pressure at point A is 350 kilopascals (kPa).  Determine
%
%\begin{enumerate}[(a)]
%\item The added head supplied by the pump station.
%\item The water pressure at B.
%\item The mechanical power supplied by the pump.
%\end{enumerate}
%
%\begin{figure}[htbp] %  figure placement: here, top, bottom, or page
%   \centering
%   \includegraphics[width=4in]{p219.pdf} 
%   \caption{Reservoir-Pump-Transmission System Schematic}
%   \label{fig:p219}
%\end{figure}
%\clearpage
%~\\

%%%%%%%%%%%%%%%%%%%%%%%%%%%%%%%%%%%%%%%%%%%%%%%%%%%%%%%%%%%%%%%%%%%%%%%%%%%%%%
%\item Figure \ref{fig:water_network_layout} is a layout of a water distribution system for the subdivision.   
%The blue line segments are pipes and are labeled (P1, P2, $\dots$).   
%The blue circles are nodes and are labeled (N1, N2, $\dots$).
%The yellow polygons represent the demand lots assigned to each node.  
%For example, node N2 supplies the six (6) individual lots located near the node.
%\begin{figure}[ht!] %  figure placement: here, top, bottom, or page
%   \centering
%   \includegraphics[width=6.5in]{SomewhereClipNodes.jpg} 
%   \caption{Water Distribution (Skeleton) System.}
%   \label{fig:water_network_layout}
%\end{figure}
%\begin{enumerate}[a)]
%\item Determine the length of each pipe in the sketch.
%\item Select an appropriate material for each pipe using San Marcos, Texas water system design guidelines and determine an appropriate Hazen-William's loss coefficient for each pipe.
%\end{enumerate}
%Use your values to produce a completed version of Table \ref{tab:ByPipes}.
%Save the table (in Excel or something similar) -- you will need it later in the design project RP-1.
%% Requires the booktabs if the memoir class is not being used
%\begin{table}[h!]
%   \centering
%   \caption{Pipe Properties for Somewhere USA Distribution System}
%   \begin{tabular}{p{1in}p{1in}p{1in}p{1in}p{1in}p{1in}} % Column formatting, @{} suppresses leading/trailing space
%Pipe ID & L (feet) & D (inches) & Material & $C_H$ & $k_s$ (inches) \\
%\hline
%\hline
%P1 & 44 & 8 & Ductile Iron & 130 & 0.0024 \\
%P2 & 440 & 2 &Ductile Iron & 130 & 0.0024 \\
%P3 & 385 & 2 &Ductile Iron & 130 & 0.0024  \\
%P4 & $\dots$ & 2 &Ductile Iron & 130 & 0.0024  \\
%P5 & $\dots$ & 2 &Ductile Iron & 130 & 0.0024  \\
%P6 & $\dots$ & 2 &Ductile Iron & 130 & 0.0024  \\
%P7 & $\dots$ & 2 &Ductile Iron & 130 & 0.0024  \\
%P8 & $\dots$ & 2 &Ductile Iron & 130 & 0.0024  \\
%P9 & $\dots$ & 2 &Ductile Iron & 130 & 0.0024  \\
%P10 & $\dots$ & 2 &Ductile Iron & 130 & 0.0024  \\
%P11 & $\dots$ & 2 &Ductile Iron & 130 & 0.0024  \\
%P12 & $\dots$ & 2 &Ductile Iron & 130 & 0.0024  \\
%P13 & $\dots$ & 2 &Ductile Iron & 130 & 0.0024  \\
%P14 & $\dots$ & 2 &Ductile Iron & 130 & 0.0024 \\
%P15 & $\dots$ & 2 &Ductile Iron & 130 & 0.0024  \\
%%P16 & ~ & ~ & ~ & ~ & ~ \\
%%P17 & ~ & ~ & ~ & ~ & ~ \\
%%P18 & ~ & ~ & ~ & ~ & ~ \\
%%P19 & ~ & ~ & ~ & ~ & ~ \\
%%P20 & ~ & ~ & ~ & ~ & ~ \\
%%P21 & ~ & ~ & ~ & ~ & ~ \\
%%P22 & ~ & ~ & ~ & ~ & ~ \\
%%P23 & ~ & ~ & ~ & ~ & ~ \\
%%P24 & ~ & ~ & ~ & ~ & ~ \\
%%P25 & ~ & ~ & ~ & ~ & ~ \\
%%P26 & ~ & ~ & ~ & ~ & ~ \\
%$\dots$ & $\dots$ & $\dots$& $\dots$&$\dots$& $\dots$ \\
%$\dots$ & $\dots$ & $\dots$& $\dots$&$\dots$& $\dots$ \\
%$\dots$ & $\dots$ & $\dots$& $\dots$&$\dots$& $\dots$ \\
%%P27 & ~ & ~ & ~ & ~ & ~ \\
%%P28 & ~ & ~ & ~ & ~ & ~ \\
%%P29 & ~ & ~ & ~ & ~ & ~ \\
%%P30 & ~ & ~ & ~ & ~ & ~ \\
%%P31 & ~ & ~ & ~ & ~ & ~ \\
%%P32 & ~ & ~ & ~ & ~ & ~ \\
%%P33 & ~ & ~ & ~ & ~ & ~ \\
%%P34 & ~ & ~ & ~ & ~ & ~ \\
%%P35 & ~ & ~ & ~ & ~ & ~ \\
%%P36 & ~ & ~ & ~ & ~ & ~ \\
%%P37 & ~ & ~ & ~ & ~ & ~ \\
%%P38 & ~ & ~ & ~ & ~ & ~ \\
%%P39 & ~ & ~ & ~ & ~ & ~ \\
%%P40 & ~ & ~ & ~ & ~ & ~ \\
%%P41 & ~ & ~ & ~ & ~ & ~ \\
%%P42 & ~ & ~ & ~ & ~ & ~ \\
%%P43 & ~ & ~ & ~ & ~ & ~ \\
%%P44 & ~ & ~ & ~ & ~ & ~ \\
%%P45 & ~ & ~ & ~ & ~ & ~ \\
%%P46 & ~ & ~ & ~ & ~ & ~ \\
%P57 & $\dots$ & 2 &Ductile Iron & 130 & 0.0024  \\
%P58 & $\dots$ & 2 &Ductile Iron & 130 & 0.0024  \\
%P59 & $\dots$ & 2 &Ductile Iron & 130 & 0.0024  \\
%P60 & $\dots$ & 2 &Ductile Iron & 130 & 0.0024  \\
%P61 & $\dots$ & 2 &Ductile Iron & 130 & 0.0024  \\
%P62 & $\dots$ & 2 &Ductile Iron & 130 & 0.0024  \\
%P63 & $\dots$ & 2 &Ductile Iron & 130 & 0.0024  \\
%P64 & $\dots$ & 2 &Ductile Iron & 130 & 0.0024  \\
%P65 & $\dots$ & 2 &Ductile Iron & 130 & 0.0024  \\
%P66 & $\dots$ & 2 &Ductile Iron & 130 & 0.0024  \\
%\hline
%   \end{tabular}
%
%   \label{tab:ByPipes}
%\end{table}

\end{enumerate}
\clearpage
\begin{thebibliography}{}

\bibitem[\protect\citeauthoryear{Swamee and Jain}{Swamee and Jain}{1976}]{jain1976}
Swamee and Jain, A. K., 1976. Explicit equations for pipe-flow problems.  ASCE J. of Hyd. Div., 102(HY5) pp. 657-664 


\end{thebibliography}


\end{document}  