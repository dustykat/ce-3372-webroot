\documentclass[12pt]{article}
\usepackage{geometry}                % See geometry.pdf to learn the layout options. There are lots.
\geometry{letterpaper}                   % ... or a4paper or a5paper or ... 
%\geometry{landscape}                % Activate for for rotated page geometry
\usepackage[parfill]{parskip}    % Activate to begin paragraphs with an empty line rather than an indent
\usepackage{daves,fancyhdr,natbib,graphicx,dcolumn,amsmath,lastpage,url}
\usepackage{amsmath,amssymb,epstopdf,longtable}
\usepackage{paralist} 
\usepackage[final]{pdfpages}
\DeclareGraphicsRule{.tif}{png}{.png}{`convert #1 `dirname #1`/`basename #1 .tif`.png}
\pagestyle{fancy}
\lhead{CE 3372 -- Water Systems Design}
%%%%%%%%%%%%%%%%%%%%%%%%%%%%%%
%%%%%%%%%%%%%%%%%%%%%%%%%%%%%%
%%%%%%%%%%%%%%%%%%%%%%%%%%%%%%
\rhead{FALL 2025}
\lfoot{EXERCISE 2}
\cfoot{}
\rfoot{Page \thepage\ of \pageref{LastPage}}
\renewcommand\headrulewidth{0pt}
%%%%%%%%%%%%%%%%%%%%%%%%%%%%%%%
%%%%%%%%%%%%%%%%%%%%%%%%%%%%%%%
%%%%%%%%%%%%%%%%%%%%%%%%%%%%%%%
\begin{document}
\begin{center}
{\textbf{{ CE 3372 -- Water Systems Design} \\ {Exercise 2}}}
\end{center}
\section*{\small{Purpose}}
\begin{itemize}
\item Demonstrate ability to summarize engineering literature, and communicating that summary in writing.
\end{itemize}
\section*{\small{Los Angeles Aqueduct History Article Review - Version 1 (Human)}}

The article presents a brief history of the Los Angeles Aquaduct as built in the early 20-th century. The system captures water from snowmelt in the Owens Valley on the Eastern Sierra Nevada range and transports that water to Los Angeles. The system is gravity flow (in that the Owens Valley is at considerably greater altitude than Los Angeles), and contains several electricity generating stations. Most of the system is open channel in either open trapezoidal channels or large-bore tunnels dug through rock. Some portions are pressurized pipes that cross valleys as inverted siphons.

The project itself involved construction of roads, rail extensions, concrete plants, and mining operations. It was, for its time, a huge engineering undertaking involving nearly all aspects of civil engineering, including land acquisition by both purchase and the legal concept of imminent domain.

The techniques in engineering and construction were later used in the construction of large dams in the Western USA, including Boulder (Hoover) Dam, and Grand Coulee Dam, which arguably contributed to the economic sucess of the USA since the 1930s.

The aqueduct is a living example that encompasses most existing water transmission technologies; except for pumping. The aqueduct has pressurized pipeline portions, turbines (energy recovery), open channel portions, and storage components. When compared to a modern gas pipeline the principle differences are the working fluid, and the requirement in a gas pipeline of compressor (pumping) stations to maintaining the driving force to transmit the product. Otherwise in terms of scale, complexity, and importance the two are equivalent.
\clearpage
\section*{\small{Los Angeles Aqueduct History Article Review - Version 2 (Computer Assisted)}}

The Los Angeles Aqueduct, completed in 1913 under the direction of William Mulholland, was a monumental engineering project that transformed Los Angeles into a modern metropolis. Designed to address the city's growing demand for water, the aqueduct stretches 233 miles from the Owens Valley to Los Angeles, transporting up to 260 million gallons of water per day. By relying entirely on a gravity-fed system, the aqueduct eliminated the need for pumps, significantly reducing operational energy costs.

The aqueduct's construction incorporated innovative techniques and infrastructure. Engineers utilized unlined and concrete-lined canals, closed conduits, and inverted siphons to navigate diverse terrains, including steep mountains and arid deserts. The Elizabeth Tunnel, a record-breaking feat at 5 miles long, showcased the use of advanced tunneling methods and incentives for rapid excavation. Additionally, steel pipe siphons, such as the Jawbone Siphon, allowed water to traverse deep canyons without interruption. Hydroelectric plants along the aqueduct not only powered construction activities but also generated electricity for Los Angeles.

Despite its engineering success, the aqueduct faced significant social and environmental challenges. The diversion of water from Owens Valley devastated local agriculture and ecosystems, leading to decades of legal disputes and public opposition. While the aqueduct enabled the rapid growth of Los Angeles, it did so at the expense of communities and environments along its route.

\section*{\small{Comparison with Large-Diameter Gas Pipelines}}

The Los Angeles Aqueduct and large-diameter gas pipelines share similarities in their purpose and design while differing in operational requirements and environmental impacts. Both systems are linear infrastructure projects that transport resources over long distances, requiring careful planning to navigate challenging terrains. They rely on durable materials—concrete and steel for aqueducts, and high-strength steel for pipelines—to withstand internal and external forces.

However, the differences between the two systems are significant. The aqueduct's gravity-fed design relies on natural elevation changes to maintain flow, whereas gas pipelines use compressor stations to pressurize and transport gas. This distinction makes the aqueduct more energy-efficient during operation but less adaptable to varying demands. In contrast, pipelines require continuous energy input but offer greater control over flow rates and distribution.

Maintenance challenges also differ. Aqueducts must address sediment buildup, erosion, and potential water loss from seepage or evaporation. Gas pipelines face risks of corrosion, pressure fluctuations, and leaks, which can pose safety hazards. Environmental impacts vary as well; the aqueduct's diversion of water caused long-term ecological and social consequences in Owens Valley, while gas pipelines risk localized habitat disruption and, in extreme cases, catastrophic explosions.

\section*{\small{Conclusion}}

The Los Angeles Aqueduct stands as a landmark in engineering history, demonstrating the transformative power of infrastructure. Its gravity-fed design and innovative construction techniques provided a sustainable water supply for Los Angeles, albeit with significant social and environmental costs. When compared to large-diameter gas pipelines, the aqueduct exemplifies a more energy-efficient system but highlights the importance of balancing technological advancement with environmental stewardship and community well-being. This analysis underscores the complexities of large-scale infrastructure projects and the need for thoughtful design and management to address both technical and societal challenges.



\end{document}  